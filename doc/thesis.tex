\title{Completion of Backwards Confluence in Context-Free Hyperedge Replacement Grammars}
\author{Johannes Schulte}
\date{\today}

\documentclass[12pt]{article}

\usepackage{amsthm}

\newtheorem{theorem}{Theorem}

\theoremstyle{definition}
\newtheorem{lemma}{Lemma}
\newtheorem{definition}{Definition}


\begin{document}
\maketitle

\begin{abstract}
%TODO
In general hypergraph transformation systems confluence and backwards confluence is not decidable. This thesis explores if it is possible to decide backwards confluence for context free hyperedge replacement grammars (HRG).
Additionally it shows heuristics to complete HRGs by adding additional rules that lead to a backwards confluent HRG.
\end{abstract}

\section{Definitions reformulated for context free HRGs}
\begin{definition}[Track morphism]
%TODO
\end{definition}

\begin{definition}[Independence]
Two direct abstractions $H \Rightarrow_{e, X \rightarrow K} G \Leftarrow_{e', X' \rightarrow K'} H' $ are independent if the two embeddings $emb$ and $emb'$ corresponding to these abstractions fulfill the following condition:
$$emb_V(K) \cap emb'_V(K') = \emptyset$$
This means the internal nodes of K and K' embedded into G are disjoint.

In this case both abstractions can be applied one after another and still lead to the same hypergraph regardless of the order in which the abstractions are applied.
\end{definition}

\begin{definition}[Backward Confluence of $G \in HRG_{\Sigma_N}$]
A HRG $G$ is backward confluent (locally backward confluent) if for all $G, G_1, G_2 \in HG_\Sigma$, $G_1 \Rightarrow^* G \Leftarrow^* G_2$ ($G_1 \Rightarrow G \Leftarrow G_2$) implies that there are $H_1, H_2 \in HG_\Sigma$ such that $G_1 \Leftarrow^*H_1 \cong H_2 \Rightarrow^* G_2$.
\end{definition}

\begin{definition}[Critical Pair]

\end{definition}

\section{Important theorems}

\begin{theorem}[Church-Rosser Theorem]
%TODO
\end{theorem}

\begin{lemma}[Newman's Lemma]
A terminating rewriting system is confluent if it is locally confluent.
\end{lemma}


\end{document}