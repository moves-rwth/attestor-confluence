\title{Backwards Confluence of Context-Free Hyperedge Replacement Grammars}
\author{Johannes Schulte}
\date{\today}

\documentclass[12pt]{article}

\usepackage{amsthm}
\usepackage{todonotes}

\setlength{\marginparwidth}{3.5cm}
\theoremstyle{definition}
\newtheorem{theorem}{Theorem}
\newtheorem{lemma}{Lemma}
\newtheorem{definition}{Definition}

\begin{document}
\maketitle

\begin{abstract}
In general hypergraph transformation systems confluence and backwards confluence is not decidable. This thesis explores if it is possible to decide backwards confluence for \emph{context-free} hyperedge replacement grammars (HRGs).
Additionally it shows heuristics to complete HRGs by adding additional rules that lead to a backwards confluent HRG.
\end{abstract}

\section{Current TODO List}
\begin{itemize}
\item Find out why strong joinability implies local confluence (Plu05)
\item Find out if counterexample (why simple joinability of critical pairs is not enough) can be modified to fit our HRGs
\item Implement critical pair generation algorithm
\end{itemize}

\section{Definitions reformulated for HRGs}
\begin{definition}[Independence]
Two direct abstractions $H \Rightarrow_{e, X \rightarrow K} G \Leftarrow_{e', X' \rightarrow K'} H' $ are independent if the two embeddings $emb$ and $emb'$ corresponding to these abstractions fulfill the following condition:
$$(emb_V(K) \cap emb'_V(K')) \setminus (ext \cap ext') = \emptyset$$
This means the internal nodes of K and K' embedded into G are disjoint.

In this case both abstractions can be applied one after another and still lead to the same hypergraph regardless of the order in which the abstractions are applied. \todo{Is this even correct for non injective rules? (External nodes mapping to same node)}
\end{definition}

\begin{definition}[Backward Confluence of $G \in HRG_{\Sigma_N}$]
A HRG $G$ is backward confluent (locally backward confluent) if for all $G, G_1, G_2 \in HG_\Sigma$, $G_1 \Rightarrow^* G \Leftarrow^* G_2$ ($G_1 \Rightarrow G \Leftarrow G_2$) implies that there are $H_1, H_2 \in HG_\Sigma$ such that $G_1 \Leftarrow^*H_1 \cong H_2 \Rightarrow^* G_2$.
\end{definition}

\begin{definition}[Critical Pair]
Two direct abstractions $U \Rightarrow_{e, X \rightarrow K} S \Leftarrow_{e', X' \rightarrow K'} U'$ is a critical pair if given the embeddings $emb$ and $emb'$ of the abstractions the following holds:
\begin{itemize}
\item $V_S = emb_V(V_K) \cup emb'_V(V_{K'})$
\item The steps are not independent.
\end{itemize}
\end{definition}

\begin{definition}[Joinability]
A critical pair $\Gamma: U \Rightarrow S \Leftarrow U'$ is \emph{joinable} if there are abstractions $U \Leftarrow^* X$ and $U' \Leftarrow^* X'$ such that $X \cong X'$. \\
$\Gamma$ is \emph{strongly joinable} if such abstractions still exist if we declare the persistent vertices as external in S. The persistent vertices are $(ext \setminus int') \cup (ext' \setminus int)$.
\end{definition}

\section{Important theorems}

\begin{theorem}[Church-Rosser Theorem]
In confluent systems the order in which reduction rules are applied does not matter.
\end{theorem}

\begin{lemma}[Newman's Lemma]
A terminating rewriting system is confluent if it is locally confluent.
\end{lemma}

\begin{lemma}[Critical Pair Lemma]
A HRG is locally backwards-confluent if all its critical pairs are strongly joinable.
\end{lemma}

\end{document}